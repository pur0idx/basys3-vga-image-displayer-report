\section{How to use \& FAQ}
\subsection{Notes}
This is a \textbf{unofficial} template, I simply made this for myself and decided that others could benefit from it. It is set up in accordance with the Uni verity of Lincolns presentation guidelines.If you have any issues using this template you can send me an email at joepittsy@gmail.com. 

\subsection{How do I edit the title page?}
In the main.tex file you will see a collection of commands such as \textbackslash yourName\{Name\} simply change these and hit recompile

\subsection{How do I change the headers/footers?}
The headers and footers will automatically update using the variables you set for the title page. If you wish to change them further you will find them in the setup.tex file in the preamble folder.

\subsection{How do I reference?}
\begin{figure}[!htb]
    \begin{verbatim}
    @misc{fridge1948,
        title={The rise of the fridge},
        journal={Telegraph.co.uk},
        author={Wilson, Bee},
        year={2011},
        note = "London: The Telegraph.
                Available from: \url{https://www.telegraph.co.uk/foodanddrink/8605239/
                                     The-rise-of-the-fridge.html}
                [Accessed 4 April 2020]"
    }
    \end{verbatim}
    \caption{Example Website Reference}
    \label{fig:refExample}
\end{figure}

Due to the University's modified referencing style this is a little difficult, if you are referencing an article then you can include the bibtex reference as usual, if it's a website you're referencing you need to do a little work. You need to include a note in your reference that includes the Place and Date of publication then the URL where you can assess the website and finally the date when you accessed it, an example of this can be seen in figure \ref{fig:refExample}. To inset an in text reference do \textbackslash citep\{fridge1948\}\citep{fridge1948}

